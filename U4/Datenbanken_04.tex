\documentclass[
	10pt,								% globale Schriftgröße
	parskip=half-,						% setzt Absatzabstand hoch
	paper=a4,							% Format
	english,ngerman,					% lädt Sprachpakete
	]{scrartcl}							% Dokumentenklasse

% //////////////////// Pakete laden ////////////////////
\usepackage{amsmath}			% MUSS vor fontspec geladen werden
\usepackage{mathtools}			% modifiziert amsmath
\usepackage{amssymb}			% mathematische symbole, für \ceckmarks
\usepackage{amsthm}				% für proof
\usepackage{mathrsfs}			% für \mathscr
\usepackage{latexsym}
\usepackage{marvosym}				% für Lightning

\usepackage{fontspec} 			% funktioniert nur mit den neueren Compilern z.B. XeLaTeX
\usepackage{microtype}			% für bessere Worttrennung
\usepackage[ngerman]{babel} 	% Spracheinstellung
\usepackage{lmodern}			% verändert verwendete Schriftart, damit sie weniger pixelig ist

\usepackage{verbatim}
\usepackage{listings}			% Für Quellcode

\usepackage{graphicx}
%\usepackage{subcaption}
\usepackage{subfig}
\usepackage{tabularx}			% für Tabellen mit gleicher Spaltenbreite und automatischen Umbrüchen
\usepackage{fullpage}
\usepackage{multirow}			% für multirow in tabulars
\usepackage{rotate}
\usepackage[cmyk,table]{xcolor} % um Farben zu benutzen, kann mehr als das Paket color
\usepackage[					% Verlinkungen
	colorlinks,					% farbige Schrift, statt farbiger Rahmen
	linktocpage,				% verlinkt im Abb.Verzeichnis Seitenzahl statt Bildunterschrift
	linkcolor=blue				% setzt Farbe der Links auf blau
	]{hyperref}					% nur für digitale Anwendungen, url = "http://www.example.com"
\usepackage{url}				% für Webadressen wie e-mail usw.: "\url{http://www.example.com}"

\usepackage{enumerate}			% für versch. Aufzählungezeichen wie z.B. a)
\usepackage{xspace}				% folgt ein Leerzeichen nach einem \Befehl, wird es nicht verschluckt.
\usepackage{cancel}				% für das Durchstreichen u.a. in Matheformeln mit \cancel
\usepackage{float}              % zum Forcieren der Position von figure-Umgebungen

% zum Zeichnen (u.a. von Graphen)
\usepackage{fp}
\usepackage{tikz}
\usetikzlibrary{tikzmark}			% für \tikzmark{toRemember}
\usetikzlibrary{positioning}	% verbesserte Positionierung der Knoten
\usetikzlibrary{automata}		% für Automaten (GTI)
\usetikzlibrary{arrows}
\usetikzlibrary{shapes}
\usetikzlibrary{decorations.pathmorphing}
\usetikzlibrary{decorations.pathreplacing}
\usetikzlibrary{decorations.shapes}
\usetikzlibrary{decorations.text}

% //////////////////// Syntaxhighlighting ////////////////////
\lstloadlanguages{Python, Haskell, [LaTeX]TeX, Java}
\lstset{
   basicstyle=\footnotesize\ttfamily,	% \scriptsize the size of the fonts that are used for the code
   backgroundcolor = \color{bgcolour},	% legt Farbe der Box fest
   breakatwhitespace=false,	% sets if automatic breaks should only happen at whitespace
   breaklines=true,			% sets automatic line breaking
   captionpos=t,				% sets the caption-position to bottom, t for top
   commentstyle=\color{codeblue}\ttfamily,% comment style
   frame=single,				% adds a frame around the code
   keepspaces=true,			% keeps spaces in text, useful for keeping indentation
							% of code (possibly needs columns=flexible)
   keywordstyle=\bfseries\ttfamily\color{codepurple},% keyword style
   numbers=left,				% where to put the line-numbers;
   							% possible values are (none, left, right)
   numberstyle=\tiny\color{codegreen},	% the style that is used for the line-numbers
   numbersep=5pt,			% how far the line-numbers are from the code
   stepnumber=1,				% nummeriert nur jede i-te Zeile
   showspaces=false,			% show spaces everywhere adding particular underscores;
							% it overrides 'showstringspaces'
   showstringspaces=false,	% underline spaces within strings only
   showtabs=false,			% show tabs within strings adding particular underscores
   flexiblecolumns=false,
   tabsize=1,				% the step between two line-numbers. If 1: each line will be numbered
   stringstyle=\color{orange}\ttfamily,	% string literal style
   numberblanklines=false,				% leere Zeilen werden nicht mitnummeriert
   xleftmargin=1.2em,					% Abstand zum linken Layoutrand
   xrightmargin=0.4em,					% Abstand zum rechten Layoutrand
   aboveskip=2ex, 
}

\lstdefinestyle{py}{
   language=Python,
}
\lstdefinestyle{hs}{
   language=Haskell,
}
\lstdefinestyle{tex}{
	language=[LaTeX]TeX,
	escapeinside={\%*}{*)},     % if you want to add LaTeX within your code
	texcsstyle=*\bfseries\color{blue},% hervorhebung der tex-Schlüsselwörter
	morekeywords={*,$,\{,\},\[,\],lstinputlisting,includegraphics,
	rowcolor,columncolor,listoffigures,lstlistoflistings,
	subsection,subsubsection,textcolor,tableofcontents,colorbox,
	fcolorbox,definecolor,cellcolor,url,linktocpage,subtitle,
	subject,maketitle,usetikzlibrary,node,path,addbibresource,
	printbibliography},% if you want to add more keywords to the set
     numbers=none,
     numbersep=0pt,
     xleftmargin=0.4em,
}

\lstdefinestyle{java}{
	language=Java,
	extendedchars=true,		% lets you use non-ASCII characters;
   						% for 8-bits encodings only, does not work with UTF-8
}

\lstdefinelanguage[x64]{Assembler}     % add a "x64" dialect of Assembler
   [x86masm]{Assembler} % based on the "x86masm" dialect
   % with these extra keywords:
   {morekeywords={CDQE,CQO,CMPSQ,CMPXCHG16B,JRCXZ,LODSQ,MOVSXD, %
                  POPFQ,PUSHFQ,SCASQ,STOSQ,IRETQ,RDTSCP,SWAPGS, %
                  rax,rdx,rcx,rbx,rsi,rdi,rsp,rbp, %
                  r8,r8d,r8w,r8b,r9,r9d,r9w,r9b}
}					% for 8-bits encodings only, does not work with UTF-8

\lstdefinestyle{c}{
	language=c,
	extendedchars=true,		% for 8-bits encodings only, does not work with UTF-8
}

% //////////////////// eigene Kommandos ////////////////////
\newcommand\FU{Freie Universität Berlin\xspace}% benötigt package xspace
\newcommand\gdw{g.\,d.\,w.\xspace}
\newcommand\oBdA{o.\,B.\,d.\,A.\xspace}
\newcommand{\Eu}{\texteuro}
\newcommand\N{\mathbb{N}\xspace}
\newcommand\Q{\mathbb{Q}\xspace}
\newcommand\R{\mathbb{R}\xspace}
\newcommand\Z{\mathbb{Z}\xspace}
\newcommand\ohneNull{\ensuremath{\backslash\lbrace 0\rbrace}}% \{0}
\let\dhALT\dh	% Schreibt Befehl \dh in \dhALT um
\renewcommand\dh{d.\,h.\xspace}	%renew überschreibt command \dh
\newcommand\Bolt{\;\text{\LARGE\raisebox{-0.3em}{\Lightning}\normalsize}\xspace}% Blitz
\newcommand\zz{\ensuremath{\raisebox{+0.25ex}{Z}% zu zeigen
			\kern-0.4em\raisebox{-0.25ex}{Z}%
			\;\xspace}}
\newcommand{\from}{\ensuremath{\colon}}
\newcommand{\floor}[1]{\lfloor{#1}\rfloor}
\newcommand{\ceil}[1]{\lceil{#1}\rceil}
 \renewcommand{\L}{\ensuremath{\mathcal{L}}\xspace}
 \renewcommand{\P}{\ensuremath{\mathcal{P}}\xspace}
 \newcommand{\NL}{\ensuremath{\mathcal{N}\kern-0.2em\mathcal{L}}\xspace}
 \newcommand{\NP}{\ensuremath{\mathcal{NP}}\xspace}

% //////////////////// Mathefunktionen ////////////////////
\DeclareMathOperator{\Landau}{\mathcal{O}}
\DeclareMathOperator{\True}{True}
\DeclareMathOperator{\False}{False}

% //////////////////// eigene Theoreme ////////////////////
\newtheorem{theorem}{Satz}
\newtheorem{corollary}[theorem]{Folgerung}
\newtheorem{lemma}[theorem]{Lemma}
\newtheorem{observation}[theorem]{Beobachtung}
\newtheorem{definition}[theorem]{Definition}
\newtheorem{Literatur}[theorem]{Literatur}
% konfiguriert proof
\makeatletter
\newenvironment{Proof}[1][\proofname]{\par
  \pushQED{\qed}%
  \normalfont \topsep6\p@\@plus6\p@\relax
  \trivlist
  \item[\hskip\labelsep
%         \itshape
        \bfseries
    #1\@addpunct{.}]\ignorespaces
}{%
  \popQED\endtrivlist\@endpefalse
}
\makeatother

% //////////////////// eigene Farben ////////////////////
\let\definecolor=\xdefinecolor
\definecolor{FUgreen}{RGB}{153,204,0}
\definecolor{FUblue}{RGB}{0,51,102}

\definecolor{middlegray}{rgb}{0.5,0.5,0.5}
\definecolor{lightgray}{rgb}{0.8,0.8,0.8}
\definecolor{orange}{rgb}{0.8,0.3,0.3}
\definecolor{azur}{rgb}{0,0.7,1}
\definecolor{yac}{rgb}{0.6,0.6,0.1}
\definecolor{Pink}{rgb}{1,0,0.6}

\definecolor{bgcolour}{rgb}{0.97,0.97,0.97}
\definecolor{codegreen}{rgb}{0,0.6,0}
\definecolor{codegray}{rgb}{0.35,0.35,0.35}
\definecolor{codepurple}{rgb}{0.58,0,0.82}
\definecolor{codeblue}{rgb}{0.4,0.5,1}

% //////////////////// eigene Settings ////////////////////

\textheight = 230mm		% Höhe des Satzspiegels / Layouts
\footskip = 10ex			% Abstand zw. Fußzeile und Grundlinie letzter Textzeile
\parindent 0pt			% verhindert Einrückung der 1. Zeile eines Absatzes
\setkomafont{sectioning}{\rmfamily\bfseries}% setzt Ü-Schriften in Serifen, {disposition}

\usepackage[backend=biber]{biblatex} %biblatex mit biber laden
\ExecuteBibliographyOptions{
sorting=nyt, %Sortierung Autor, Titel, Jahr
bibwarn=true, %Probleme mit den Daten, die Backend betreffen anzeigen
isbn=false, %keine isbn anzeigen
url=true
}											% bindet Header ein (WICHTIG)

\newcommand{\dozent}{Prof. Dr. Agnès Voisard \\ Nicolas Lehmann}					% <-- Names des Dozenten eintragen
\newcommand{\tutor}{Toni Draßdo}						% <-- Name eurer Tutoriun eintragen
\newcommand{\tutoriumNo}{014}				% <-- Nummer im KVV nachschauen
\newcommand{\ubungNo}{04}									% <-- Nummer des Übungszettels
\newcommand{\veranstaltung}{Datenbanksysteme}	% <-- Name der Lehrveranstaltung eintragen
\newcommand{\semester}{SoSe 18}						% <-- z.B. SoSo 17, WiSe 17/18
\newcommand{\studenten}{Eduard Beiline, Mark Niehues, Antoen Oehler}			% <-- Hier eure Namen eintragen
\addbibresource{./src/lit.bib} %Bibliographiedateien laden

\usepackage[normalem]{ulem}

% /////////////////////// BEGIN DOKUMENT /////////////////////////
\begin{document}
% /////////////////////// BEGIN TITLEPAGE /////////////////////////
\begin{titlepage}
	\subject{\dozent}
	\title{\veranstaltung, \semester}
	\subtitle{\Large Übung \ubungNo\\ \large\vspace{1ex} TutorIn: \tutor\\ Tutorium \tutoriumNo}
	\author{\studenten}
	\date{\normalsize \today}
\end{titlepage}

\maketitle								% Erstellt das Titelblatt
\vspace*{-10cm}							% rückt Logo an den oberen Seitenrand
\makebox[\dimexpr\textwidth+1cm][r]{	%rechtsbündig und geht rechts 1cm über Layout hinaus
	\includegraphics[width=0.4\textwidth]{../src/fu_logo} % fügt FU-Logo ein
}
% /////////////////////// END TITLEPAGE /////////////////////////

\vspace{7cm}							% Abstand
\rule{\linewidth}{0.8pt}				% horizontale Linie										% erstellt die Titelseite



\section*{Task 1: ER-Modellierung}


\section*{Task 2: Relationales Modell}


\section*{Task 3: Reverse Engineering}

\section*{Task 4: Data Mining}
\subsection*{1 - K-Means}
Der Log des K-Mean Aufrufs ist in Listing \ref{log} angegeben. Der dazugehörige entwickelte Code in Listing \ref{code}.

\lstinputlisting[							% Style
	caption={Log File des K-Means Algorithmus},		% Beschriftung
	firstnumber={1},										% Start der Nummerierung
	firstline={0},
	label = {log}											% 1. Codezeile
]											% letzte Codezeile
{./src/4_1_output.txt}

\lstinputlisting[style=py,								% Style
	caption={K-Means Implementierung},		% Beschriftung
	firstnumber={1},										% Start der Nummerierung
	firstline={0},
	label={code}										% 1. Codezeile
]											% letzte Codezeile
{./src/4_1.py}

\subsection*{2 - Naive Bayes}
\subsubsection*{1 - Wahrscheinlichkeit einer Grippe bei laufender Nase}
Naive Bayes Formula:
\begin{equation}
P(C|x) = \frac{P(C)\, P(x|C)}{P(x)}
\label{bayes}
\end{equation}
Aus Formel \ref{bayes} folgt für die Wahrscheinlichkeit an einer Grippe zu leiden, bei laufender Nase:

\begin{equation}
P(Grippe|Nase) = \frac{P(Gripp)\, P(Nase|Grippe)}{P(Nase)}
\label{eq1}
\end{equation}
wobei:

\begin{align*}
&P(Nase) = 4/8 = 1/2 \\
&P(Grippe) = 1/2 \\
&P(Nase|Grippe) = 3/4
\end{align*}

Durch einsetzen in Formel \ref{eq1} erhält man:

\begin{equation*}
P(Grippe|Nase) = 3/4
\end{equation*}

\subsubsection*{2 - Grippe, wenn X}
Um die Frage zu beantwortet, ob jemand eher Grippe oder keine Grippe besitzt wird, um die Rechnung zu Vereinfachen der Quotient aus $P(Grippe|x)$ und $P(\neg Grippe|x)$ gebildet, dadurch kürzt sich die aufwändig zu berechnende Evidenz $P(x)$ heraus:

\begin{equation}
Q = \frac{P(Grippe|x)}{P(\neg Grippe|x)} = \frac{P(x|Grippe) \, P(Grippe)}{P(x|\neg Grippe) \, P(\neg Grippe)}
\end{equation}

wobei:

\begin{align*}
& x = \{Schüttelfrost, schwache Kopfschmerzen, Fieber\} \\
& P(Schüttelfrost|Grippe) = 3/4 \\
& P(Schüttelfrost|\neg Grippe) = 1/2 \\
& P(schwache Kopfschmerzen|Grippe) = 1/4 \\
& P(schwache Kopfschmerzen|\neg Grippe) = 1/4 \\
& P(Fieber|Grippe) = 1/2 \\
& P(Fieber|\neg Grippe) = 1/2 \\
& P(Grippe) = P(\neg Grippe) = 1/2 \\
\end{align*}
Unter Annahme der (hinreichenden) Unabhängigkeit der Variablen, gilt $P(x_1, x_2 |C) = P(x_1|C)\, P(x_2|C)$.

Daraus ergibt sich schließlich:
\begin{equation*}
Q = \frac{\frac{3}{4} \, \frac{1}{4}\, \frac{1}{2}}{\frac{1}{2} \, \frac{1}{4}\, \frac{1}{2}} = 3/2
\end{equation*}

Aus der $Q > 1$ folgt, dass der Patient wahrscheinlicher Grippe hat als keine.

\subsection*{3 - Apriori}
\def\arraystretch{1.5}%  1 is the default, change whatever you need
\begin{table}[h]
\centering
\caption{Ausschnitt aus der Berechnung der Supports durch Kombinatorik und Abzählen}
\begin{tabular}{l | l | l | l | l}
$C_0$ & $L_0$ & $C_1$ & $L_1$ & $C_2$ \\ \hline
sup(\{A\}) = $\frac{1}{2}$ & \{A\} & sup(\{A,B\}) = $\frac{1}{2}$ & \{A,B\} & ... \\
sup(\{B\}) = $\frac{5}{6}$ & \{B\} & sup(\{A,C\}) = $\frac{1}{3}$ & \{A,E\} & ... \\
sup(\{C\}) = $\frac{1}{2}$ & \{C\} & sup(\{A,D\}) = $\frac{1}{3}$ & \{B,C\} & ... \\
sup(\{D\}) = $\frac{5}{6}$ & \{D\} & sup(\{A,E\}) = $\frac{2}{3}$ & \{B,D\} & ... \\
sup(\{E\}) = $\frac{5}{6}$ & \{E\} & sup(\{A,F\}) = $\frac{1}{6}$ & \{B,E\} & ... \\
sup(\{F\}) = $\frac{1}{2}$ & \{F\} & sup(\{B,C\}) = $\frac{1}{2}$ & \{C,D\} & ... \\
& & sup(\{B,D\}) = $\frac{2}{3}$ & \{D,E\} & ... \\
& & sup(\{B,E\}) = $\frac{2}{3}$ & \{D,F\} & ... \\
& & sup(\{B,F\}) = $\frac{1}{3}$ & \{E,F\} & ... \\
& & sup(\{C,D\}) = $\frac{1}{2}$ & & ... \\
& & sup(\{C,E\}) = $\frac{1}{3}$ & & ... \\
& & sup(\{C,F\}) = $\frac{1}{6}$ & & ... \\
& & sup(\{D,E\}) = $\frac{2}{3}$ & & ... \\
& & sup(\{D,F\}) = $\frac{1}{2}$ & & ... \\
& & sup(\{E,F\}) = $\frac{1}{2}$ & & ... \\
\end{tabular}
\end{table}

Die Supports werden anschließend wiederum kombiniert um Beziehungen der Form $\{A,B\} \rightarrow \{C\}$ zu bewerten. Die sogenannte \textit{confidence} berechnet sich aus den Support als:
\begin{equation}
conf(X \rightarrow Y) = \frac{sup(X \cup Y)}{sup(X)}
\end{equation}

Das Ergebnis der Berechnungen sei an dieser Stelle der Einfachheit halber als Textdatei angegeben:
\lstinputlisting[								% Style
	caption={Ergebnis des Apriori Algorithmus.},		% Beschriftung
	firstnumber={1},										% Start der Nummerierung
	firstline={0}
]											% letzte Codezeile
{./src/4_3_output.txt}

\subsection*{4 - Lineare Regression}
Leider ist bei uns erst eine Korrektur eingetragen, daher nehmen wir für diese Aufgabe folgende Noten an:
\begin{table}[h]
\centering
\caption{Bisherige Notenverteilung}
\begin{tabular}{l | c c c}
x & 0 & 1 & 2 \\ \hline
y & 0,89 & 0,92 & 0,93 \\
\end{tabular}
\label{noten}
\end{table}

Aus Tabelle \ref{noten} folgt: $\bar{x} = 1$ und $\bar{y} = 0.91\bar{3}$.

Die lineare Regression beschreibt die Daten mit einer Funktion der Art:
\begin{equation}
f(x) = \beta _0 + \beta _1 \, x + \epsilon
\label{lin}
\end{equation}
mit
\begin{align}
\beta _1 & = \frac{\sum \limits_{i=1}^n (x_i - \bar{x}) \, (y - \bar{y})}{\sum \limits_{i=1}^n (x_i - \bar{x})^2} \label{beta0}\\
\beta _0 & = \bar{y} - \beta _1 \, \bar{x} \label{beta1}\\
\end{align}

Aus Gleichung \ref{beta0} folgt $\beta _1 = 0,02$ und damit $\beta _0 = 0.893$.
Aus der linearen Gleichung \ref{lin} ergibt sich also für den nächsten Zettel $f(3) = 0,953$.
\printbibliography %hier Bibliographie ausgeben lassen

\end{document}
